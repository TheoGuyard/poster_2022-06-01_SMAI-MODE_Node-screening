%%%%%%%%%%%%%%%%%%%%%%%%%%%%%%%%%%%%%%%%%
% Jacobs Landscape Poster
% LaTeX Template
% Version 1.1 (14/06/14)
%
% Created by:
% Computational Physics and Biophysics Group, Jacobs University
% https://teamwork.jacobs-university.de:8443/confluence/display/CoPandBiG/LaTeX+Poster
% 
% Further modified by:
% Nathaniel Johnston (nathaniel@njohnston.ca)
%
% This template has been downloaded from:
% http://www.LaTeXTemplates.com
%
% License:
% CC BY-NC-SA 3.0 (http://creativecommons.org/licenses/by-nc-sa/3.0/)
%
%%%%%%%%%%%%%%%%%%%%%%%%%%%%%%%%%%%%%%%%%

%----------------------------------------------------------------------------------------
%	PACKAGES AND OTHER DOCUMENT CONFIGURATIONS
%----------------------------------------------------------------------------------------

\documentclass[final]{beamer}

\usepackage[scale=1.24]{beamerposter}
\usepackage{tikz}
\usetikzlibrary{matrix}
\usetikzlibrary{positioning}
\usetikzlibrary{er}
\usepackage{ragged2e}

\usetheme{confposter}

\setbeamercolor{block title}{fg=dblue,bg=white}
\setbeamercolor{block body}{fg=black,bg=white}
\setbeamercolor{block alerted title}{fg=white,bg=dblue!70}
\setbeamercolor{block alerted body}{fg=black,bg=dblue!10}
\setbeamercolor{item}{fg=nblue}
\setbeamercolor{item projected}{fg=white,bg=nblue}

\newlength{\sepwid}
\newlength{\onecolwid}
\newlength{\twocolwid}
\newlength{\threecolwid}
\setlength{\paperwidth}{46.8in} % A0 width: 46.8in
\setlength{\paperheight}{33.1in} % A0 height: 33.1in
\setlength{\sepwid}{0.024\paperwidth} % Separation width (white space) between columns
\setlength{\onecolwid}{0.22\paperwidth} % Width of one column
\setlength{\twocolwid}{0.464\paperwidth} % Width of two columns
\setlength{\threecolwid}{0.708\paperwidth} % Width of three columns
\setlength{\topmargin}{-1.5in} % Reduce the top margin size

\usepackage{kmath}
% ------------------------ %
% 	  Packages		%
% ------------------------ %

\usepackage[utf8]{inputenc}
\usepackage[T1]{fontenc}
\usepackage{xcolor}
\usepackage{kmath}
\usepackage{xparse,pgffor,ifthen,xspace}
\usepackage{amsmath,amssymb,amsfonts,amsthm}
\usepackage{mathrsfs,mathtools,bbm}
\usepackage{booktabs,dcolumn,multirow,stmaryrd}
\newcolumntype{d}{D{.}{.}{3.2}}
\newcolumntype{e}{D{.}{.}{1.2}}
\usepackage{pgfplots,forest}
\pgfplotsset{compat=newest}
\usepackage{cleveref,csquotes,comment,lipsum,todonotes,soul}
\usepackage{glossaries,url,subcaption}
\include{headings/shortcuts}
\include{headings/macros}
\usepackage[pscoord]{eso-pic}
\newcommand{\placetextbox}[3]{
  \setbox0=\hbox{#3}
  \AddToShipoutPictureFG*{
    \put(\LenToUnit{#1\paperwidth},\LenToUnit{#2\paperheight}){\vtop{{\null}\makebox[0pt][c]{#3}}}
  }
}

\usepackage[ruled]{algorithm2e}

\newcommand{\emphone}[1]{{\color{norange}#1}}

\newacronym{bnb}{BnB}{Branch-and-Bound}
\newacronym{mip}{MIP}{Mixed-Integer Program}
\newcommand{\METHOD}{NODE-SCREENING\xspace}
\newcommand{\Method}{Node-screening\xspace}
\newcommand{\method}{node-screening\xspace}
\newcommand{\sparsitylevel}{k}
\newcommand{\sparsevector}{\pv^0}
\newcommand{\CPLEX}{\texttt{Direct}}
\newcommand{\BNB}{\texttt{BnB}}
\newcommand{\BNBscr}{\texttt{BnB+scr}}

%----------------------------------------------------------------------------------------
%	TITLE SECTION 
%----------------------------------------------------------------------------------------

\title{Node-screening tests for the $\mathbf{\ell}_{\0}$-penalized least-squares problem}

\author{Théo Guyard${}^{\star}$, Cédric Herzet${}^{\dagger}$, Clément Elvira${}^{\ddagger}$}

\institute{${}^{\star}$Applied Mathematics Department, INSA Rennes, France | ${}^{\dagger}$SIMSMART team, INRIA Rennes-Bretagne Atlantique, France | ${}^{\ddagger}$SCEE team, CentraleSupelec Rennes, France}

%----------------------------------------------------------------------------------------

\begin{document}

%\placetextbox{0.955}{0.995}{\fbox{Paper ID : 2482}}

\addtobeamertemplate{block end}{}{\vspace*{2ex}}
\addtobeamertemplate{block alerted end}{}{\vspace*{2ex}}

\setlength{\belowcaptionskip}{2ex}
\setlength\belowdisplayshortskip{2ex}

\begin{frame}[t]
\begin{columns}[t]

\begin{column}{\sepwid}\end{column}

\begin{column}{\onecolwid}

    \setbeamercolor{block alerted title}{fg=white,bg=norange}
    \setbeamercolor{block alerted body}{fg=black,bg=white}
    \begin{alertblock}{Objectives}
        Reduce the optimization time of a \gls{bnb} algorithm tailored to the $\ell_0$-penalized least-squares problem by detecting nodes of the decision tree that cannot yield a global optimizer.
    \end{alertblock}

    \begin{block}{Introduction}
        \begin{itemize}
            \item \hspace*{0.1em} \justifying \textbf{Sparse decomposition:} Find an approximation of a vector $\obs$ as the \emph{linear combination} of a \emph{few} columns of a matrix $\dic$.
        \end{itemize}
        %
        \vspace*{0.5em}
        %
        \setbeamercolor{block alerted title}{fg=white,bg=dblue!70}
        \setbeamercolor{block alerted body}{fg=black,bg=dblue!10}
        \begin{alertblock}{$\boldsymbol{\ell}_{\0}$-penalized least-squares}
            \begin{tikzpicture}
                \node (basis)[minimum width=3em] {};
                \node (norm2) [left=2em of basis.north west] {$\norm{\obs-\dic\pv}{2}$};
                \node (datafidelity) [below=0em of norm2] {Ensures \emphone{data-fidelity}};
                \node (norm0) [right=2em of basis.north east] {$\norm{\pv}{0}$};
                \node (sparsity)[below=0em of norm0] {Promotes \emphone{sparsity}};
                \node (problem) [below=3.5em of basis.north] {
                    $
                    \hspace*{-3em}
                    (\mathcal{P}) \quad
                    \left\{
                    \begin{array}{rl}
                        \min & \ \tfrac{1}{2} \norm{\obs-\dic\pv}{2}^2 + \reg \norm{\pv}{0} \\
                        \text{st.} & \ \norm{\pv}{\infty} \leq \bigM
                    \end{array}
                    \right.
                    $
                };
                \draw[thick,->,line width=5pt] (datafidelity) -- (problem);
                \draw[thick,->,line width=5pt] (sparsity) -- (problem);
            \end{tikzpicture}
        \end{alertblock}
        %
        \begin{itemize}
            \item \hspace*{0.1em} \textbf{Characteristics:}
            \begin{itemize}
                \normalsize \item[-] \hspace*{0.1em} NP-hard problem due to the $\ell_0$-norm
                \item[-] \hspace*{0.1em} \glsdesc{mip} reformulation
                \item[-] \hspace*{0.1em} Addressable with \gls{bnb} algorithms  
                \item[-] \hspace*{0.1em} ``Big-M'' to construct bounded relaxations
            \end{itemize}
        \end{itemize}

        \begin{itemize}
            \item \hspace*{0.1em} \textbf{Recent advances by Atamturk \textit{et. al.}:}
            \begin{itemize}
                \normalsize \justifying \item[-] \hspace*{0.1em} \emphone{Screening tests} to detect zeros and non-zeros in the optimizers of the problem
                \item[-] \hspace*{0.1em} Dimensionality reduction in \emphone{pre-processing}
            \end{itemize}
        \end{itemize}

        \begin{itemize}
            \item \hspace*{0.1em} \textbf{Our contributions:}
            \begin{itemize}
                \normalsize \justifying \item[-] \hspace*{0.1em} \emphone{Node-screening tests} to detect zeros and non-zeros in the optimizers of \emphone{any node} problem
                \item[-] \hspace*{0.05em} Dimensionality reduction at \emphone{any step} of the optimization process
                \item[-] \hspace*{0.05em} Larger improvement in the solving time
            \end{itemize}
        \end{itemize}
    \end{block}
\end{column}

\begin{column}{\sepwid}\end{column}

\begin{column}{\twocolwid}    
    \begin{columns}[t,totalwidth=\twocolwid]
        \begin{column}{\onecolwid}\vspace{-.6in} 
            
            \begin{block}{\gls{bnb} algorithms}
                \begin{itemize}
                    \item \hspace*{0.1em} \textbf{Generic procedure:}
                \end{itemize}
                \vspace*{1em}
                \input{algo} 
                \vspace*{1em}
                \begin{itemize}
                    \item \hspace*{0.1em} \textbf{Efficiency of the algorithm:}
                    \begin{itemize}
                        \normalsize 
                        \item[-] \hspace*{0.1em} Tightness of the bounds computed
                        \item[-] \hspace*{0.1em} Ability to process nodes quickly
                        \item[-] \hspace*{0.1em} \emphone{Number of nodes processed}
                    \end{itemize}
                \end{itemize}
                \begin{itemize}
                    \item \hspace*{0.1em} \textbf{Particularized to our problem:}
                    \begin{itemize}
                        \normalsize \justifying \item[-] \hspace*{0.1em} Binary tree with decision about the \emphone{nullity of an entry} at each node
                        \item[-] \hspace*{0.1em} Each node is uniquely defined by the set of entries forced to \emphone{zero} and to \emphone{non-zero}
                    \end{itemize}
                \end{itemize}
            \end{block}
        \end{column}

        \begin{column}{\onecolwid}\vspace{-.6in}
            \begin{block}{Node-screening tests}
                \begin{itemize}
                    \item \hspace*{0.1em} \textbf{Underlying idea:}
                    \begin{itemize}
                        \normalsize \justifying 
                        \item[i)] \hspace*{0.1em} For many nodes, it is obvious that they cannot yield a global solution (too large penalty, \textit{etc}...)
                        \item[ii)] \hspace*{0.1em} \gls{bnb} is an exact method so we have to \emphone{prove} that they can be pruned safely
                        \item[iii)] \hspace*{0.1em} Standard pruning methodology : solve a convex relaxation. This is \emphone{expensive} !
                        \item[iv)] \hspace*{0.1em} Why not trying \emphone{weaker} but \emphone{cheaper} methods ? 
                    \end{itemize}
                \end{itemize}
                \begin{center}
                    \begin{tikzpicture}[
                        every node/.style = 
                            {
                                draw=dblue!70,
                                fill=dblue!10,
                                thick,
                                text width=0.45\linewidth,
                                align=center,
                                line width=5pt,
                            }
                        ]
                        \node (hard) {``Non-obvious'' nodes};
                        \node (easy) [right=of hard] {``Obvious'' nodes};
                        \node (hardmethod) [below=1.5em of hard] {Standard pruning};
                        \node (easymethod) [below=1.5em of easy] {Node-screening};
                        \draw[thick,->,line width=5pt] (hard) -- (hardmethod);
                        \draw[thick,->,line width=5pt] (easy) -- (easymethod);
                    \end{tikzpicture}
                \end{center}
                \vspace*{0.5em}
                \begin{itemize}
                    \item \hspace*{0.1em} \textbf{Differences with the classical method:}
                    \begin{itemize}
                        \normalsize \item[-] \hspace*{0.1em} Test \emphone{any} unfixed index, no branching rule
                        \item[-] \hspace*{0.1em} No computational overhead to prune nodes
                        \item[-] \hspace*{0.1em} Can fix \emphone{multiple variables simultaneously}
                    \end{itemize}
                \end{itemize}
            \end{block}
            \begin{alertblock}{Main ingredients}
                \begin{itemize}
                    \item \hspace*{0.1em} At each node, a \emphone{convex relaxation} is solved
                    \item \hspace*{0.1em} We construct the \emphone{dual} of this relaxation
                    \item \hspace*{0.1em} Duals are very similar between two nodes
                    \item \hspace*{0.1em} At node $\nu$, we use this relation to hopefully \emphone{prune the child node} of $\nu$ constructed by branching on a given index
                \end{itemize}
            \end{alertblock}
        \end{column}
    \end{columns}

    \begin{figure}
        \begin{subfigure}[b]{0.49\textwidth}
            \centering
            \forestset{
    node/.style = {
        draw, 
        circle,
        thick,
        align = center,
        font = \small,
        top color = white,
        bottom color = blue!25,
        line width=5pt,
    },
    branch label/.style={
        edge label = {
            node[midway,fill=white,font=\small,draw=black,text height=0.5em,scale=0.9]{#1}
        }
    },
    bnb/.style={
        branch label,
        for tree = {
            node,
            s sep'+=35mm,
            l sep'+=10mm,
            edge={->},
            edge+={thick,line width=5pt},
        },
        before typesetting nodes={
            for tree={
                split option={content}{:}{content,branch label},
            },
        },
        where n children=0{
            tikz+={
                \draw [thick,dashed,->,line width=5pt]  ([yshift=0pt, xshift=0pt].south east) -- ([yshift=-15pt, xshift=15pt].south east);
                \draw [thick,dashed,->,line width=5pt]  ([yshift=0pt, xshift=0pt].south west) -- ([yshift=-15pt, xshift=-15pt].south west);
            }
        }{},
    },
}
\begin{forest}
    bnb,
    [
        \(\nodeSymbIter{0}\)
            [\(\nodeSymbIter{1}\):\({\pve_{\idxentrynode_1}=0}\)
            [\(\nodeSymbIter{3}\):\({\pve_{\idxentrynode_2}=0}\)]
            [\(\nodeSymbIter{4}\):\({\pve_{\idxentrynode_2}\neq0}\)]
    ][
        \(\nodeSymbIter{2}\):\({\pve_{\idxentrynode_1}\neq0}\),
            [\(\nodeSymbIter{5}\):\({\pve_{\idxentrynode_2}=0}\)]
            [\(\nodeSymbIter{6}\):\({\pve_{\idxentrynode_2}\neq0}\)]
    ]
    ]
\end{forest}
        \end{subfigure}
        \hfill
        \begin{subfigure}[b]{0.49\textwidth}
            \centering
            \forestset{
    node/.style = {
        draw, 
        circle,
        thick,
        align = center,
        font = \small,
        top color = white,
        bottom color = blue!25,
        line width=5pt,
    },
    branch label/.style={
        edge label = {
            node[midway,fill=white,font=\small,draw=black,text height=0.5em,scale=0.9]{#1}
        }
    },
    bnb/.style={
        branch label,
        for tree = {
            node,
            s sep'+=35mm,
            l sep'+=10mm,
            edge={->, thick, draw opacity=0.2, text opacity = 0.3,line width=5pt},
            opacity = 0.25,
            text opacity = 0.25,
        },
        before typesetting nodes={
            for tree={
                split option={content}{:}{content,branch label},
            },
        },
        where n children=0{
            tikz+={
                \draw [thick,dashed,->,line width=5pt,opacity=0.25]  ([yshift=0pt, xshift=0pt].south east) -- ([yshift=-15pt, xshift=15pt].south east);
                \draw [thick,dashed,->,line width=5pt,opacity=0.25]  ([yshift=0pt, xshift=0pt].south west) -- ([yshift=-15pt, xshift=-15pt].south west);
            }
        }{},
    },
}
\begin{forest}
    bnb,
    [\(\nodeSymbIter{0}\),name=root,opacity=1,text opacity=1
        [\(\nodeSymbIter{1}\):\({\pve_{\idxentrynode_1}=0}\)
            [\(\nodeSymbIter{3}\):\({\pve_{\idxentrynode_2}=0}\),name=S3,opacity=1,text opacity=1] {
                \draw [thick,dashed,->,line width=5pt] ([yshift=0pt, xshift=0pt].south east) -- ([yshift=-15pt, xshift=15pt].south east);
                \draw [thick,dashed,->,line width=5pt]  ([yshift=0pt, xshift=0pt].south west) -- ([yshift=-15pt, xshift=-15pt].south west);
                \draw[norange,thick,->,line width=5pt] (root.west) .. controls (-11.5,0) .. (S3.north);
                \node[draw=norange,thick,top color = white,bottom color = orange!25,font=\small,align=center,line width=5pt] at (-7,1) (scr) {\bf {\small \Method} \\ Fix \({\pve_{\idxentrynode_1}=0}\) \\ Fix \({\pve_{\idxentrynode_2}=0}\)};
            }
            [\(\nodeSymbIter{4}\):\({\pve_{\idxentrynode_2}\neq0}\)]
        ]
        [\(\nodeSymbIter{2}\):\({\pve_{\idxentrynode_1}\neq0}\),
            [\(\nodeSymbIter{5}\):\({\pve_{\idxentrynode_2}=0}\)]
            [\(\nodeSymbIter{6}\):\({\pve_{\idxentrynode_2}\neq0}\)]
        ]
    ]
\end{forest}
        \end{subfigure}
    \end{figure}
\end{column}

\begin{column}{\sepwid}\end{column}

\begin{column}{\onecolwid}
    \begin{block}{Numerical results}
        \begin{itemize}
            \item \hspace{0.1in} \textbf{Data generation:}
            \begin{itemize}
                \justifying
                \normalsize \item[1)] \hspace{0.1in} Set $(\ddim,\pdim)=(500,1000)$
                \item[2)] \hspace{0.1in} Generate a random matrix $\dic \in \kR^{\ddim\times\pdim}$ with a correlation $\rho$ between the columns
                \item[3)] \hspace{0.1in} Generate a $k$-sparse vector $\pv^{\dagger} \in \kR^{\pdim}$
                \item[4)] \hspace{0.1in} \normalsize Set $\obs = \dic\pv^{\dagger} + \text{noise with 10dB SNR}$
                \item[5)] \hspace{0.1in} \normalsize Calibrate $\lambda$ and $M$ statistically to recover $\pv^{\dagger}$
            \end{itemize}
        \end{itemize}
        \begin{itemize}
            \item \hspace{0.1in} \textbf{Concurrent methods:}
            \begin{itemize}
                \normalsize \item[-] \hspace{0.1in} \normalsize Direct method using CPLEX
                \item[-] \hspace{0.1in} \normalsize Tailored \gls{bnb} algorithm from Mhenni \textit{et. al.} 
                \item[-] \hspace{0.1in} \normalsize Tailored \gls{bnb} algorithm with \emphone{node-screening}
            \end{itemize}
        \end{itemize}
        \setlength{\tabcolsep}{11pt}

\begin{table}[!ht]
	\small
	\centering
    \begin{tabular}{cc||c|c|c||c|c|c||c|c|c}
	\toprule
	& & \multicolumn{3}{c||}{Direct} & \multicolumn{3}{c||}{BnB} & \multicolumn{3}{c}{BnB + node-scr.} \\
	$\rho$ & \(\sparsitylevel\) & \multicolumn{1}{c}{N} & \multicolumn{1}{c}{T} & \multicolumn{1}{c||}{F} & \multicolumn{1}{c}{N} & \multicolumn{1}{c}{T} & \multicolumn{1}{c||}{F} & \multicolumn{1}{c}{N} & \multicolumn{1}{c}{T} & \multicolumn{1}{c}{F} \\ \midrule\midrule
    & \(5\) & 0.1 & 25.9 & 0 & 0.1 & 1.5 & 0 & \textbf{0.05} & \textbf{0.7} & \textbf{0} \\
    & \(7\) & 0.3 & 60.8 & 0 & 0.2 & 5.1 & 0 & \textbf{0.1} & \textbf{3.0} & \textbf{0} \\ 
    \parbox[t]{8mm}{\multirow{-3}{*}{\rotatebox[origin=c]{90}{Low}}} & \(9\) & 0.8 & 102.6 & 10 & 0.5 & 15.6 & 0 & \textbf{0.4} & \textbf{9.8} & \textbf{0} \\ \midrule
	& \(5\) & 1.4 & 10.2 & 0 & 1.0 & 6.4 & 0 & \textbf{0.7} & \textbf{4.2} & \textbf{0} \\
    & \(7\) & 17.6 & 106.5 & 0 & 10.5 & 79.3 & 0 & \textbf{7.9} & \textbf{52.2} & \textbf{0} \\ 
    \parbox[t]{8mm}{\multirow{-3}{*}{\rotatebox[origin=c]{90}{High}}} & \(9\) & 80.7 & 353.4 & 50 & 47.8 & 346.4 & 48 & \textbf{41.2} & \textbf{267.0} & \textbf{40} \\ \bottomrule                      
    \end{tabular}
	\caption{
		\small
		\label{tab:figure}
		(N) : Nodes explored $\times10^{-3}$. (T) : Solving time in sec. (F) : Number of instances not solved within $10^3$ sec.
	}
\end{table}

        \begin{itemize}
            \item \hspace{0.1in} \textbf{Observations:}
            \begin{itemize}
                \normalsize \justifying \item[-] \hspace*{0.1in} BnB + node-scr. has the best performances
                \item[-] \hspace*{0.1in} Time gain is larger than node gain
                \item[-] \hspace*{0.1in} Double kiss-cool effect : the bounding step is performed all the faster as \emphone{many variables are fixed}
            \end{itemize}
        \end{itemize}
        \vspace*{1em}
        \setbeamercolor{block alerted title}{fg=white,bg=norange}
        \setbeamercolor{block alerted body}{fg=black,bg=white}
        \begin{alertblock}{Take home message}
            In a \gls{bnb} tailored to a sparse problem, there is not always need to perform heavy computations to prune nodes.
            Many nodes can be \emphone{easily pruned} by performing \emphone{simple and cheap tests} like the node-screening one.
        \end{alertblock}
    \end{block}
\end{column} 

\begin{column}{\sepwid}\end{column}

\end{columns}
\end{frame}

\end{document}
